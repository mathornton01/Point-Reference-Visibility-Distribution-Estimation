\documentclass{article} 
\usepackage[top=1in,bottom=1in,left=1in,right=1in]{geometry} 

\title{Fitting Prior distributions to Path Data} 
\author{Micah A. Thornton} 
\date{\today} 
\pagenumbering{gobble} 


\begin{document} 

\maketitle{} 

\abstract{Sometimes spatial data is supplied relative to a central observer.  The observer
follows a particular path, and records their location at each sighting of an event of interest, if they are diligent then they will not only record their own locations, but also the estimated distance of the observation from them, as well as an approximation of the angle from which they
make the observation.  If the position of the observer is fixed at the origin of a graph, then
these two components (angle and magnitude) can be used to form a relative spatial point process
that is relative to the observers position and orientation. This would allow for the analysis 
of a so called 'vision' distribution that represents the relative likelihood of observation given the angle and magnitude of the distance from the observer.  In this project the data-set 
that will be used relates to the latitudes and longitudes of a sea-faring vessel as it spotted 
Whales along its course.} 

\section{Introduction} 

\end{document} 